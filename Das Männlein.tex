% !TeX spellcheck = de_DE
\documentclass{article}

\usepackage[german]{babel}
\usepackage{microtype}
\usepackage{verse}
\usepackage{parskip}
\usepackage{hyperref}

\newcommand{\titlevar}{Das Männlein im Land der Giganten}
\newcommand{\authorvar}{Günthner}
\newcommand{\datevar}{Sommer 2023}
\title{\vspace{-2cm}\titlevar}
\author{\authorvar}
\date{\datevar}
\hypersetup{
	pdftitle=\titlevar,
	pdfauthor=\authorvar,
	pdfcreationdate=\datevar,
}
\setlength{\parindent}{0pt}

\begin{document}
	\maketitle
	 
	Das Männlein lebte in einem kleinen Häuschen aus Holz. Innen war dieses Häuschen schmuckvoll verziert mit kleinen filigranen Schnitzereien, allerdings blickte das Männlein nie nach Innen. Es kümmerte sich nie um das Innen. Es besaß nämlich ein schönes Teleskop, mit dem es nach draußen in das Land der Giganten sehen konnte:  
	
	\textit{Fensterscheibe; nach unten; Hohe Holzkonstruktion; Berg; Noch mehr Berg; Noch viel viel mehr Berg; Nun Wald mit weiten Pfaden; Die unendliche Ebene; Und; In der Ebene: Die Höhlen der Giganten.}
	
	Das Männlein konnte die Giganten von hier aus beobachten, was es auch den ganzen Tag über tat. Auch die ganze Nacht über versuchte es in das Leben der Giganten einzudringen, allerdings schlief es immer irgendwann am Teleskop ein und wachte erst am nächsten Mittag wieder auf. Die Giganten faszinierten das Männlein so sehr, es wünschte sich so sehr, dass es so sein könnte wie sie: Sie waren nie allein, waren immer glücklich und verspürten keinen Schmerz. Das Männlein hingegen war immer allein, nur durch Zuschauen konnte es die Einsamkeit etwas verringern. Das Männlein hingegen war auch nicht immer glücklich; Nein, vielmehr war es so wenig wie möglich. Es verbrachte seine Jahre entweder damit, zu schlafen oder beim Leben der Giganten mitzufühlen. Das Männlein hingegen verspürte auch immer Schmerz: Jedes Mal, wenn es draußen etwas schönes oder in seinem Kämmerchen etwas schreckliches erblickte. Schmerzerfüllt ging es durch seinen Alltag \dots
	 
	Eines besonders schönen Tages war das Männlein gezwungen, sein Teleskop zu reparieren, keine Ablenkung. Es versuchte, die Arbeit möglichst schnell hinter sich zu bringen. Irgendwann wurde es hungrig und legte eine Pause ein. Der Tisch war aus rohem Holz, wo man sich überall Splitter einziehen konnte. Auf dem Tisch befand sich ein unschönes Chaos: Irgendwelche Papiere, Zettel, Stifte, Essensreste, Teile des Teleskops. Auch die Linsen. Das Männlein war eine Weile weg. Was es jetzt nur schon wieder anstellte? Die Linsen lagen im Sonnenlicht. Im hellen Sonnenlicht.  
	
	\begin{verse}[5cm]
		Feuer, Feuer, flieg hinfort,  \\
		Feuer, Feuer, fetz es weg, \\ 
		Feuer, Feuer, form es neu, \\
		Feuer, Feuer, \textbf{brenn lichterloh\!}
	\end{verse}
	
	Das gesamte Kämmerchen stand in Flammen, das Männlein stand auf. Nichts konnte mehr getan werden, es gab ja nicht einmal einen Ausgang aus dem Haus.  
	
	Es gibt ja nicht einmal einen Ausgang aus dem Haus, wozu auch, es steht ja auf einer hohen Holzkonstruktion. Vielleicht könnte ich da runterklettern? Ich hatte noch nie darüber nachgedacht. Ich bewege mich in Richtung eines der Fenster, als das Haus als Gesamtwerk birst. Die hohen Stelzen, auf denen das Haus steht, der Boden, die Wände, die Decke. Und ich bin mittendrin, im Inferno. Ich stürze, alles, meine ganze Welt stürzt.
	
	Ich wache auf.  
	
	Um mich herum liegen Trümmer. Ein kaputtes Teleskop. Und ich. Brandwunden zieren meine Haut. Meine Haare sind komplett verschwunden. Meine langen, schönen Haare sind auf einmal weg. Ich fühle mich einfach nur leer. Meine Welt ist zerstört und verbrannt. Ich gehe also los. Den Berg hinunter. Ich laufe tagelang. Ich laufe wochenlang. Ich kriege Hunger. Ich habe Sehnsucht nach meinem Teleskop. Ich würde so unendlich gerne wieder einen Blick hinein werfen können, sei es auch nur für einen Moment. Mit der Zeit wird mir klar, wie einsam ich dort oben war. Wie viele Jahre ich dort oben gelitten habe, vergeudet. Mit jedem Schritt werden meine Gedanken klarer; Ich trete ein in den großen, grünen Wald. Die Bäume um mich herum sind noch größer, als mir dort oben vorkamen. Sie ragen auf in den Himmel und scheinen nie, nie, nie aufzuhören. Überall ist Grün, überall ist Leben. Schließlich führt meine Reise mich zu einem der weiten Pfaden. Er ist groß. Wie ein riesiger Fluss, einer, der ganze Seen füllen kann, einer, der große Boote tragen kann, einer, in dem Wale schwimmen können. Ich setze meine Reise fort, nun aber immer den Pfad entlang. Nach weiteren Tagen des Wanderns fällt mir ein tiefes, periodisches Grollen auf. Mit der Zeit gesellen sich immer mehr von diesen Geräuschen zur Musik des Waldes, die bis dahin nur angenehmes Vogelgezwitscher beinhaltete. Die Geräusche werden lauter. Ich kriege Angst. Ich höre auf zu Atmen, mein Blick gleitet nach oben, ich fange an, langsam rückwärts zu laufen, meine Schritte werden immer schneller, plötzlich sauge ich Luft ein. Ich renne vom Pfad tiefer in den Wald. Ich bringe einen der großen Bäume zwischen mich und  
	
	Angst  \\
	schneidet tief in mein kleines Herz.
	
	“Unendliche Weiten”, so könnte man die Giganten beschreiben. Ich kann ihre wahre Größe nicht erkennen, ich weiß nur, dass ich etwas so kolossales noch nie gesehen habe. Sind das die gleichen mystischen Wesen, die ich jeden Tag durch mein Teleskop betrachtet habe? Schließlich werde ich ohnmächtig.
	
	Ich wache auf.  
	
	Um mich herum ist immer noch der Wald. Es ist Nacht. Nicht Vollmond. Nicht Halbmond. Dunkelheit. Zwischen den Silhouetten der Bäume kann ich ein flackerndes Licht erkennen. Seine Farbe ist ein grelles Orange und ich kann erkennen, wie es tanzt; als wäre es lebendig. Es ist so wunderschön. Langsam, aber sicher bewege ich mich auf meine Bake zu. Je näher ich komme, desto klarer wird mir, dass es tatsächlich ein Lagerfeuer ist; Und dass daneben einer der Giganten schläft. Wie in Trance marschiere ich in Richtung des Wesens. Ich stehe davor. Ich stehe davor. Ich stehe vor einem Giganten. \textit{Er ist so viel mehr als ich.} Ich habe noch nie eines ihrer Gesichter gesehen. Er schläft so tief; Ich könnte wahrscheinlich… Und schon klettere ich das Ungetüm empor. Von oben ist der Gigant sogar noch schöner. So traumhaft schön. So unmöglich schön. Diese Faszination ist mir unerklärlich, aber ich könnte dieses Wesen für den Rest meines Lebens betrachten… Seine \textit{Augen schlagen auf und er begrüßt mich.} Dieses Gefühl ist unbeschreiblich—\textit{Nein}, es \textit{ist} beschreiblich: Nach all den Jahren absoluter Isolation, nach all den Jahren absoluter Akzeptanz, wird diese Grenze durchbrochen.  
	
	\begin{verse}[5cm]
		Das kleine Männlein \\ 
		hat ein kleines Herz, \\  
		das zerrissen wurde \\
		vom großen Schmerz
	\end{verse}
	
	Jedes Wort, das ich ausspreche, verbrennt meine Seele ein bisschen mehr, aber dennoch kann ich den Strom meiner Sprache nicht stoppen. Ich will ihn nicht stoppen. Ich will für alle Ewigkeit hier bleiben. Die Nacht weicht dem Tag und mein Herz beginnt zu heilen. Am Morgen erstickt der Gigant das Feuer und macht sich daran, sich zu verabschieden. Magen zieht sich zusammen. Ich kann niemanden zwingen, mir meine Einsamkeit zu nehmen. Ohnmächtig wünsche ich dem Koloss eine sichere Weiterreise. Was soll ich denn sonst tun? Was hätte ich sonst tun können? Ich blicke ihm noch lange nach. Sogar als er schon ein paar Tage nicht mehr zu sehen ist, blicke ich ihm nach. Meine Einsamkeit badet den Boden in einem Bach aus Traurigkeit. Leer setze ich meine Wanderung fort. In weiter Ferne kann ich immer wieder das dumpfe Grollen der Giganten vernehmen, aber kein einziges Mal komme ich einem von ihnen nochmal so nahe. Irgendwann lichtet sich der Wald und macht diesen ewigen Ebenen Platz. \textit{ewigen\dots}
	
	\begin{verse}[5cm]
		Ich stehe da und gehe \\
		Beweg’ mich nie vom Fleck \\
		Schreite fort und sehe \\
		Nur unter mir den Dreck \\
		
		\rule{0pt}{0pt}\\
		
		Es zieht mich aber weiter \\
		Das Glück, dass ich einst hat’ \\
		Wie auf einer Leiter \\
		Steig ich ganz tief hinab \\
		
		\rule{0pt}{0pt}\\
		
		Es nimmt einfach kein Ende \\
		Alles immer gleich			\\
		Was war, das spricht noch Bände \\
		Ich ertrinke in diesem Teich	\\
		
		\rule{0pt}{0pt}\\
		
		Das schwere kalte Wasser	\\
		Erstickt mich immer mehr	\\
		Mein Gesicht wird weiter blasser	\\
		Doch fürcht’ ich den Tod so sehr	\\
	\end{verse}
	
	So plötzlich, dass ich fast gestürzt wäre, verwandelt sich der Boden, auf den ich gestarrt hatte, in ein endloses Loch. \textit{Eine Höhle.} War mal war, kann wiederkehren. Ich steige also hinein. Es ist tief schwarz. Ich taste mich herum. Durchsuche die große Umgebung. Von dem, was ich erfühlen kann, gibt es immer wieder neue Abzweigungen und Wege. Ich durchsuche die Dunkelheit für eine Ewigkeit, finde aber nichts bis auf neue Dunkelheit. Irgendwann erblicke ich wieder den Mond, der als schmerzhafte Erinnerung meiner Vergangenheit den Himmel mit seinem \textit{Licht} erfüllt. Ich ziehe weiter, in der Hoffnung, wenigstens noch einmal jemanden zu treffen.  
	Viele Höhlen durchsuche ich noch, ohne Erfolg. Nirgendwo ein Geräusch, eine Regung irgendeiner Art. Ich ziehe weiter. Irgendwann nimmt die Müdigkeit wieder überhand.
	
	Ich wache Auf.
	
	Vor mir erstreckt sich ein Berg in die Höhe. Ein Berg aus \textit{Fleisch} und \textit{Blut}. Der plötzliche Gast umgibt mich mit seiner freundlichen Präsenz. Wieder versinke ich in einem Strom aus Sätzen. Aber es läuft gleich ab. Irgendwann macht der Gigant sich daran aufzubrechen. Ich fürchte, mein Herz wird gleich bersten. Unerwarteterweise lädt der Titan mich ein, ihn zu begleiten. In meinem Leben wurde mir noch nie ein so großes Geschenk unterbreitet. Ich nehme es an.  
	
	Wir gehen Seite an Seite. Erst einmal in Stille getaucht. Irgendwann setzt der Gigant an, etwas zu sagen. Aber dann fängt er an zu singen. Seine Stimme erforscht alle Höhen und Tiefen des Liedes, das aus seinem Herz direkt in meines zu fließen scheint. Die Genesung meiner Seele schreitet mit jedem Ton fort. Dieses Geschenk, es trifft mich so sehr, es reißt mich auseinander, es zerstört, was mal war. Dieses Mal ist es kein Bach, es ist ein Fluss, der auf die Erde niederprasselt. Irgendwann wird der Gigant, der Titan, der Riese aufhören zu singen. Der perfekte Moment, das größte Geschenk wird nur noch in der Erinnerung des Männleins existieren. Es gibt nun aber Hoffnung für das Männlein. Hoffen wir, dass es die Hoffnung sieht.
	
	Ende
	
	The Reparation of my heart…
\end{document}